% Options for packages loaded elsewhere
\PassOptionsToPackage{unicode}{hyperref}
\PassOptionsToPackage{hyphens}{url}
%
\documentclass[
]{book}
\usepackage{lmodern}
\usepackage{amsmath}
\usepackage{ifxetex,ifluatex}
\ifnum 0\ifxetex 1\fi\ifluatex 1\fi=0 % if pdftex
  \usepackage[T1]{fontenc}
  \usepackage[utf8]{inputenc}
  \usepackage{textcomp} % provide euro and other symbols
  \usepackage{amssymb}
\else % if luatex or xetex
  \usepackage{unicode-math}
  \defaultfontfeatures{Scale=MatchLowercase}
  \defaultfontfeatures[\rmfamily]{Ligatures=TeX,Scale=1}
\fi
% Use upquote if available, for straight quotes in verbatim environments
\IfFileExists{upquote.sty}{\usepackage{upquote}}{}
\IfFileExists{microtype.sty}{% use microtype if available
  \usepackage[]{microtype}
  \UseMicrotypeSet[protrusion]{basicmath} % disable protrusion for tt fonts
}{}
\makeatletter
\@ifundefined{KOMAClassName}{% if non-KOMA class
  \IfFileExists{parskip.sty}{%
    \usepackage{parskip}
  }{% else
    \setlength{\parindent}{0pt}
    \setlength{\parskip}{6pt plus 2pt minus 1pt}}
}{% if KOMA class
  \KOMAoptions{parskip=half}}
\makeatother
\usepackage{xcolor}
\IfFileExists{xurl.sty}{\usepackage{xurl}}{} % add URL line breaks if available
\IfFileExists{bookmark.sty}{\usepackage{bookmark}}{\usepackage{hyperref}}
\hypersetup{
  pdftitle={A Field Guide to Open Data},
  pdfauthor={Alicia Brown},
  hidelinks,
  pdfcreator={LaTeX via pandoc}}
\urlstyle{same} % disable monospaced font for URLs
\usepackage{longtable,booktabs}
\usepackage{calc} % for calculating minipage widths
% Correct order of tables after \paragraph or \subparagraph
\usepackage{etoolbox}
\makeatletter
\patchcmd\longtable{\par}{\if@noskipsec\mbox{}\fi\par}{}{}
\makeatother
% Allow footnotes in longtable head/foot
\IfFileExists{footnotehyper.sty}{\usepackage{footnotehyper}}{\usepackage{footnote}}
\makesavenoteenv{longtable}
\usepackage{graphicx}
\makeatletter
\def\maxwidth{\ifdim\Gin@nat@width>\linewidth\linewidth\else\Gin@nat@width\fi}
\def\maxheight{\ifdim\Gin@nat@height>\textheight\textheight\else\Gin@nat@height\fi}
\makeatother
% Scale images if necessary, so that they will not overflow the page
% margins by default, and it is still possible to overwrite the defaults
% using explicit options in \includegraphics[width, height, ...]{}
\setkeys{Gin}{width=\maxwidth,height=\maxheight,keepaspectratio}
% Set default figure placement to htbp
\makeatletter
\def\fps@figure{htbp}
\makeatother
\setlength{\emergencystretch}{3em} % prevent overfull lines
\providecommand{\tightlist}{%
  \setlength{\itemsep}{0pt}\setlength{\parskip}{0pt}}
\setcounter{secnumdepth}{5}
\usepackage{booktabs}
\ifluatex
  \usepackage{selnolig}  % disable illegal ligatures
\fi
\usepackage[]{natbib}
\bibliographystyle{apalike}

\title{A Field Guide to Open Data}
\author{Alicia Brown}
\date{2021-02-01}

\begin{document}
\maketitle

{
\setcounter{tocdepth}{1}
\tableofcontents
}
\hypertarget{welcome}{%
\chapter{Welcome!}\label{welcome}}

Thank you for joining me on a journey to explore the use of Open Data in the civic world.

The book is organized into 3 parts -

\begin{itemize}
\tightlist
\item
  Part 1 provides an overview of what Open Data is, how it is used and who uses it.
\item
  Part 2 questions the usefulness of Open Data to its producers, how use is measured and what are the challenges faced by producers and consumers of it.
\item
  Part 3 imagines how information provided by Open Data may be shared, discovered and improved.
\end{itemize}

I have had the great honor to work with many amazing organizations, agencies and governments to implement technology solutions that enabled the unleashing of data and analytics to power decision making and transparency.\footnote{\citet{blog5years}} The purpose for this book is to synthesize what I have learned working in the field with customers and present examples and ideas to grow the Open Data ecosystem.

\hypertarget{overview}{%
\chapter{An Open Data Primer}\label{overview}}

\hypertarget{what-is-open-data}{%
\section{What is Open Data?}\label{what-is-open-data}}

\begin{quote}
``Open data is information or content made freely available to use and redistribute, subject only to the requirement to attribute it to the source.''

--- Gartner\footnote{\citet{gartner_opendata_def}}
\end{quote}

\hypertarget{topics}{%
\subsection{Topics}\label{topics}}

Popular data topics include emergency calls for service, budget, revenue, expenditures, property tax, permits, food inspections, government service calls, lost and found pets, crime incidents, jail bookings, voting registration and election results. When this data includes date, time and location, it can be valuable for performance management and refining staffing for services. When it includes demographic information it makes it possible to analyze differences in data by gender, race and ethnicity, and age and can help inform where there should be more outreach and consider mode of communication that may be most effective.

\hypertarget{types}{%
\subsection{Types}\label{types}}

\begin{itemize}
\tightlist
\item
  Tabular
\item
  Spatial
\item
  Dashboards
\item
  Stories
\item
  Visualizations
\item
  Surveys
\item
  Reports
\end{itemize}

\hypertarget{how-is-it-used}{%
\section{How is it used?}\label{how-is-it-used}}

\hypertarget{organizational-goals}{%
\subsection{Organizational Goals}\label{organizational-goals}}

\begin{itemize}
\tightlist
\item
  Public information
\item
  Transparency
\item
  Fine tune staffing and location
\item
  Equity and performance measurement
\item
  Insights and data mining
\end{itemize}

\hypertarget{news}{%
\subsection{News}\label{news}}

\hypertarget{civic}{%
\subsection{Civic}\label{civic}}

\hypertarget{who-uses-it}{%
\section{Who uses it?}\label{who-uses-it}}

\begin{itemize}
\tightlist
\item
  Producer organizations
\item
  Community users (civic)
\item
  Academic
\end{itemize}

\hypertarget{personas}{%
\subsection{Personas}\label{personas}}

There are many people and roles across organizations that require data. Here are some personas based on individuals I have met and worked with to get the data they need to perform their jobs. Please note that the names are made up and intended to reflect any specific gender.

\begin{itemize}
\tightlist
\item
  Alex the Analyst
\item
  Pat the Performance Manager
\item
  the Executive
\end{itemize}

\hypertarget{use}{%
\chapter{Use of Open Data}\label{use}}

\hypertarget{examples-of-usefulness}{%
\section{Examples of Usefulness}\label{examples-of-usefulness}}

\begin{itemize}
\tightlist
\item
  Outcomes
\end{itemize}

\hypertarget{measuring-use}{%
\section{Measuring Use}\label{measuring-use}}

\begin{itemize}
\tightlist
\item
  Utilization
\item
  Analytics
\item
  Feedback
\end{itemize}

\begin{enumerate}
\def\labelenumi{\arabic{enumi}.}
\tightlist
\item
  Who are the real users of this data?
\item
  How can we tell who they are?
\item
  Are there personas we haven't imagined?
\item
  How can we measure actual engagement of these users?
\item
  Are they using it the way we thought they were? Ex. Hack to ``fix'' data in between steps.
\end{enumerate}

\hypertarget{challenges-of-open-data}{%
\section{Challenges of Open Data}\label{challenges-of-open-data}}

\hypertarget{searching-for-data}{%
\subsection{Searching for data}\label{searching-for-data}}

\begin{itemize}
\tightlist
\item
  Paging through results
\item
  Onus of filtering
\end{itemize}

\hypertarget{peeking-at-data}{%
\subsection{Peeking at data}\label{peeking-at-data}}

Nobody wants to download 1 million records of taxi rides, but they do want to know why tips increased after a software update.\footnote{\citet{iquantny2020}}

\hypertarget{imagine}{%
\chapter{Imagining Open Data}\label{imagine}}

\hypertarget{sharing}{%
\section{Sharing}\label{sharing}}

Adding datasets to a platform may benefit power users of data and make it easy consumable by tools and programming languages. However data alone will not advance the knowledge of the community it is intended to serve without also including narrative and insights from the collections of datasets shared to a platform.

\begin{itemize}
\tightlist
\item
  Stories \& Narrative
\item
  Live tiles
\item
  Data driven documentation
\item
  Dashboards
\item
  Goals
\item
  Reports
\item
  Info sheets
\end{itemize}

\hypertarget{accessing}{%
\section{Accessing}\label{accessing}}

\begin{itemize}
\tightlist
\item
  Tools
\item
  Open source library
\item
  APIs
\end{itemize}

\hypertarget{discovery}{%
\section{Discovery}\label{discovery}}

\begin{itemize}
\tightlist
\item
  Metadata
\item
  Harvestable
\item
  Standard schemas
\end{itemize}

\hypertarget{improving-the-ecosystem}{%
\section{Improving the Ecosystem}\label{improving-the-ecosystem}}

\begin{itemize}
\tightlist
\item
  Ownership and stewardship
\item
  Automation
\item
  Data refreshes
\item
  Quality monitoring
\end{itemize}

  \bibliography{book.bib,packages.bib}

\end{document}
